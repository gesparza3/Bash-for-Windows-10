\documentclass[12pt, titlepage]{article}

\usepackage[margin=0.5in]{geometry}
\usepackage{graphicx}

\title{Python Installation for Windows 10}
\author{Grant Esparza}
	
\begin{document}
	
	\section*{Getting Ready}
	
		In order to install the bash shell command-line tool, you must have the \textbf{Windows 10 Creators Update} installed.  In order to check if your PC meets the requirements:
			
			\begin{enumerate}
				
				\item Open \textbf{Settings}.
				
				\item Click on \textbf{System}.
				
				\item Click on \textbf{About}.
				
			\end{enumerate} 
		Under version, you must have at least version \textbf{1703}.  If not, you must update your PC.
		
	\section*{Installing Bash shell command-line tool for Windows 10}
	
		\begin{enumerate}
			
			\item Open \textbf{Settings.}
			
			\item Click on \textbf{Update \& security.}

			\item Click on \textbf{For Developers.}

			\item Under "Use developer features", select the Developer mode option to setup the environment to install Bash.		
			
			\item After all the components are installed, you will need to restart your computer.
		
			\item Next, open the \textbf{Control Panel.}
			
			\item Click on \textbf{Programs.}
			
			\item Select \textbf{Turn Windows features on or off.}
			
			\item Scroll down in the popup window and check the box for \textbf{Windows Subsystem for Linux(Beta)}.  
				
				\begin{itemize}
					\item \textit{On more recent versions of Windows, this feature is no longer beta.}
				\end{itemize}
						
			\item More components will install on your computer, after which select the popup \textbf{Restart Now.}
			
			\item After reboot, click \textbf{Start} and search for \textbf{\textit{bash.exe}}.  Press enter or click to open the bash window.
			
			\item The command prompt will then open.  Press 'y' to install bash.  You will then be prompted to create a UNIX account.  \textbf{Remember this info!} 			
				
				\begin{itemize}
					
					\item \textit{If your \textbf{Windows Subsystem for Linux} was not marked as beta, you will receive the following message:}	
										
						\textbf{\textit{Windows subsystem for Linux has no installed distributions. Distributions can be installed by visiting the Windows Store: https://aka.ms/wslstore	}}					
					
					Open the \textbf{Windows Store}. Search for and install \textbf{Ubuntu.} Repeat step 11. 
		
				\end{itemize}
		
		\end{enumerate}
		
	\section*{Using the Bash shell}
	
		\begin{itemize}				
				
			\item To access the bash shell tool, simply open a command prompt and type
			\textbf{bash}.  While it isn't designed to run Linux graphical applications, most
			UNIX commands will work letting you run programs in a Linux environment without
			the overhead of a virtual machine.  
			
			\item The tool's home directory is placed where it was installed so I would recommend
			adding an alias called \textbf{home} to making navigate through your directory
			eaiser.  To do so, use the command \textbf{vim \~/.bashrc} and add this line to
			the end of the file:
		
				\begin{verbatim}
					alias home="cd /mnt/c/Users/[Your Windows Username]"
				\end{verbatim}
		
			\item From now on you can simply type "\textbf{home}" on the command-line to be
			redirected to your home directory				
				
		\end{itemize}
					
	\section*{Installing Python}
				
		\begin{itemize}
			
			\item Check to see if python3 is already installed using:
				
				\begin{verbatim}
					python3 --version
				\end{verbatim}			
			
			\item If it is, great! Else, use the command:
					
					\begin{verbatim}
						sudo apt-get install python3
					\end{verbatim}		
			
		\end{itemize}
		
	Your should have everything you need to get started.		
			
	\end{document}